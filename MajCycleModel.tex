\documentclass[useAMS,usenatbib,referee]{article}
\usepackage{amsmath}
\usepackage{microtype}
\usepackage{mathptmx}
\usepackage{graphicx}
\usepackage{booktabs}
\usepackage{captcont}
\usepackage[colorlinks=true,citecolor=blue]{hyperref}
\usepackage{emaxima}
\usepackage{natbib}

\newcommand{\unit}[1]{\mathrm{#1}}
\newcommand{\unitp}[2]{\ensuremath{\mathrm{#1}^{#2}}}

\title{Computational Requirements for SDP\\
  Rev: \input{|"git describe --dirty"}}
\author{B. Nikolic\\
  Astrophysics Group, Cavendish Laboratory, Cambridge CB3 0HE, UK
  \\\url{email:b.nikolic@mrao.cam.ac.uk}
 \\\url{http://www.mrao.cam.ac.uk/~bn204/}}

\begin{document}

\maketitle

\tableofcontents

\section{Scope of this document}

This document estimates the computational requirements for the SDP for
each of the SKA telescopes.  The estimate is currently provisional and
subject to revision due to both improvements in our knowledge and
inevitable discovery of errors of method or incorrect approximations
in the calculations.

We currently only consider the computational cost of the `major
cycle', i.e., the formation of dirty sky image from the visibilities
and the reverse step of estimating model visibilities from a trial sky
brightness distribution. We do not take into account `minor-cycle'
(i.e, image-plane deconvolution), flagging, calibration, point source
finding or other data processing.

\section{Description of algorithms}

We proceed on assumption that the estimation of both the trial sky
brightness distribution (`back step') and of the model visibilities
(`forward step') will include correction of both A and $w$ terms by
convolution in the $uv$ plane. See \cite{Cornwell2008-4703511},
\cite{Hymphreys132}, \cite{2012SPIE.8500E..0LC},
\cite{2008A&A...487..419B}. The size of $w$ kernels are analysed by
\cite{Hymphreys132} and \cite{Mitchell2014}. In general case
$w$-kernels can be very large and so for this reason it is assumed a
technique for controlling the $w$ term like snapshot or
$w$-stacking. The $w$-snapshots approach is analysed in most detail.

In current version of this computation each snapshot is assumed to
require one FFT and one re-projection. The costs of computing PSFs etc
are not accounted for. The relation for duration of snapshots is given
below.


\section{Computations}

\subsection{Telescope information}

We use the following basic information about the telescopes. This is
based on \cite{DewdneyDD001-1} with its addendum \cite{McCoolDD003}.
No baseline dependent averaging is assumed.
\begin{maxima}[]
SKA1Low : [ tdump = 0.6 `s, Na = 1024 , Nf= 256000, Nbeam=1, Ds=35`m,
lambda = constvalue(%c) / (100e6`Hz) , Bmax = 100e3`m, NAA=9];
SKA1Mid : [ tdump = 0.08 `s, Na = 190+64 , Nf= 256000, Nbeam=1, Ds=15`m,
lambda=constvalue(%c) / (1160e6`Hz) , Bmax = 200e3`m, NAA=9];
SKA1Survey: [tdump = 0.3`s, Na=96, Nf=256000, Nbeam=36,      Ds=15`m,
lambda=constvalue(%c) / (1160e6`Hz) , Bmax = 50e3`m, NAA=9];
\maximaoutput*
\m  \left[ t_{\rm dump}=0.6\;\mathrm{s} , N_{\rm a}=1024 , N_{\rm f}=256000 , N_{\rm beam}=1 , D_{\rm s}=35\;\mathrm{m} , \lambda=2.9\;{{\mathrm{m}}\over{\mathrm{s}\,\mathrm{Hz}}} , B_{\rm max}=100000.\;\mathrm{m} , N_{\rm AA}=9 \right] \\
\m  \left[ t_{\rm dump}=0.08\;\mathrm{s} , N_{\rm a}=254 , N_{\rm f}=256000 , N_{\rm beam}=1 , D_{\rm s}=15\;\mathrm{m} , \lambda=0.2\;{{\mathrm{m}}\over{\mathrm{s}\,\mathrm{Hz}}} , B_{\rm max}=200000.\;\mathrm{m} , N_{\rm AA}=9 \right] \\
\m  \left[ t_{\rm dump}=0.3\;\mathrm{s} , N_{\rm a}=96 , N_{\rm f}=256000 , N_{\rm beam}=36 , D_{\rm s}=15\;\mathrm{m} , \lambda=0.2\;{{\mathrm{m}}\over{\mathrm{s}\,\mathrm{Hz}}} , B_{\rm max}=50000.\;\mathrm{m} , N_{\rm AA}=9 \right] \\
\end{maxima}
Here $\lambda$ is a representative wavelength for calculation below.
All the calculations are made at single wavelength rather than being
integrated over the bands of the telescope.

\subsection{Image size}

The sizes of images to be produced can be estimated as follows:

\begin{maxima}[]
QPix:  2 ;
NPix : QPix * 2 * Bmax/ Ds / Nfacet;

calcTel(NPix, SKA1Low);
calcTel(NPix, SKA1Mid);
calcTel(NPix, SKA1Survey);
\maximaoutput*
\m  2 \\
\m  {{4\,B_{\rm max}}\over{D_{\rm s}\,N_{\rm facet}}} \\
\m  \mathrm{NPix(SKA1Low)}={{11000.}\over{N_{\rm facet}}} \\
\m  \mathrm{NPix(SKA1Mid)}={{53000.}\over{N_{\rm facet}}} \\
\m  \mathrm{NPix(SKA1Survey)}={{13000.}\over{N_{\rm facet}}} \\
\end{maxima}

The size of $uv$-grids is assumed to be the same. 

\subsection{Size of w-projection kernels}

The size of support of w-projection kernels is analysed by
\cite{Mitchell2014} who find that:
\begin{maxima}[]
NGWfn (w, eta )  := 2 * sqrt ('qty( ((w * (lambda/ Ds) /2))**2 + (w * (lambda/ Ds) /2 )/(3.14 *eta/sqrt(w))))  / (Ds/lambda/2);
\maximaoutput*
\m  \mathrm{NGWfn}\left(w , \eta\right)\mathbin{:=}{{2\,\sqrt{\mathrm{qty}\left(\left({{w\,\left({{\lambda}\over{D_{\rm s}}}\right)}\over{2}}\right)^2+{{{{w\,\left({{\lambda}\over{D_{\rm s}}}\right)}\over{2}}}\over{{{3.1\,\eta}\over{\sqrt{w}}}}}\right)}}\over{{{{{D_{\rm s}}\over{\lambda}}}\over{2}}}} \\
\end{maxima}
where $\eta$ is the amplitude level of the kernel where we wish to
cutoff and $w$ is the w term to be corrected. 

\subsection{Snapshot duration}

Worst case scenario is that that the baseline is rotating exactly in
$w$ direction, therefore maximum $w$ rate of deviation is:
\begin{maxima}[]
We: 7.27e-5 ` (1/s); 
DeltaW : Bmax* We/2;

calcTel(DeltaW, SKA1Low);
calcTel(DeltaW, SKA1Mid);
calcTel(DeltaW, SKA1Survey);

\maximaoutput*
\m  7.27 \times 10^{-5}\;{{1}\over{\mathrm{s}}} \\
\m  3.63 \times 10^{-5}\,B_{\rm max}\;{{1}\over{\mathrm{s}}} \\
\m  \mathrm{DeltaW(SKA1Low)}=3.6\;{{\mathrm{m}}\over{\mathrm{s}}} \\
\m  \mathrm{DeltaW(SKA1Mid)}=7.2\;{{\mathrm{m}}\over{\mathrm{s}}} \\
\m  \mathrm{DeltaW(SKA1Survey)}=1.8\;{{\mathrm{m}}\over{\mathrm{s}}} \\
\end{maxima}

Using \cite{Mitchell2014} expression for w-kernel sizes above we can
calculate size of convolution kernel as $N_{\rm GW}$ given snapshot
duration as below. Also shown below are kernel sizes for 2 minute
snapshots:
\begin{maxima}[]
Qw:1;

NGW :NGWfn ( DeltaW * twsnap / lambda,  0.01/Qw );

calcTelA(NGW, SKA1Low, [twsnap=120`s] )``1;
calcTelA(NGW, SKA1Mid, [twsnap=120`s] )``1;
calcTelA(NGW, SKA1Survey, [twsnap=120`s] )``1;


\maximaoutput*
\m  1 \\
\m  {{4\,\sqrt{\mathrm{qty}\left({{3.48 \times 10^{-6}\,B_{\rm max}\,\mathrm{twsnap}\,\sqrt{{{B_{\rm max}\,\mathrm{twsnap}}\over{\lambda}}}}\over{D_{\rm s}}}\;{{1}\over{\mathrm{s}^{{{3}\over{2}}}}}+{{3.3 \times 10^{-10}\,B_{\rm max}^2\,\mathrm{twsnap}^2}\over{D_{\rm s}^2}}\;{{1}\over{\mathrm{s}^2}}\right)}\,\lambda}\over{D_{\rm s}}} \\
\m  \mbox{{}NGW(SKA1Low,[twsnap = 120 ` s]){}}=16. \\
\m  \mbox{{}NGW(SKA1Mid,[twsnap = 120 ` s]){}}=16. \\
\m  \mbox{{}NGW(SKA1Survey,[twsnap = 120 ` s]){}}=5.6 \\
\end{maxima}

\subsection{Visibility rate}

The rate at which the SDP receives visibilities is:

\begin{maxima}[]
Nbl       :  Na * (Na-1)/2 * Nf * Nbeam* 4;
VisInRate : Nbl / tdump ;

calcTel(VisInRate, SKA1Low);
calcTel(VisInRate, SKA1Mid);
calcTel(VisInRate, SKA1Survey);
\maximaoutput*
\m  2\,\left(N_{\rm a}-1\right)\,N_{\rm a}\,N_{\rm beam}\,N_{\rm f} \\
\m  {{2\,\left(N_{\rm a}-1\right)\,N_{\rm a}\,N_{\rm beam}\,N_{\rm f}}\over{t_{\rm dump}}} \\
\m  \mathrm{VisInRate(SKA1Low)}=8.93 \times 10^{+11}\;{{1}\over{\mathrm{s}}} \\
\m  \mathrm{VisInRate(SKA1Mid)}=4.11 \times 10^{+11}\;{{1}\over{\mathrm{s}}} \\
\m  \mathrm{VisInRate(SKA1Survey)}=5.6 \times 10^{+11}\;{{1}\over{\mathrm{s}}} \\
\end{maxima} 

This takes into account 4 polarisation products and  the multiple
beams for SKA1 Survey.

\subsection{FFT Rate}

Assuming 1 FFT per snapshot per polarisation product per frequency
channel per beam we get the FLOP rate for FFTs:
\begin{maxima}[]
FFTRate : (5 `Ops) * NPix **2 * log2(NPix **2)  * Nf * Nbeam * 4 * Nfacet**2 /
twsnap ;

calcTel(FFTRate, SKA1Low);
calcTel(FFTRate, SKA1Mid);
calcTel(FFTRate, SKA1Survey);

calcTelA(FFTRate, SKA1Low, [Nfacet=1, twsnap=40`s]);
calcTelA(FFTRate, SKA1Mid, [Nfacet=1, twsnap=200`s]);
calcTelA(FFTRate, SKA1Survey, [Nfacet=1, twsnap=550`s]);


\maximaoutput*
\m  {{320\,B_{\rm max}^2\,N_{\rm beam}\,N_{\rm f}\,\log \left({{16\,B_{\rm max}^2}\over{D_{\rm s}^2\,N_{\rm facet}^2}}\right)}\over{\log 2\,D_{\rm s}^2\,\mathrm{twsnap}}}\;\mathrm{Ops} \\
\m  \mathrm{FFTRate(SKA1Low)}={{9.64 \times 10^{+14}\,\log \left({{1.3 \times 10^{+8}}\over{N_{\rm facet}^2}}\right)}\over{\mathrm{twsnap}}}\;\mathrm{Ops} \\
\m  \mathrm{FFTRate(SKA1Mid)}={{2.1 \times 10^{+16}\,\log \left({{2.84 \times 10^{+9}}\over{N_{\rm facet}^2}}\right)}\over{\mathrm{twsnap}}}\;\mathrm{Ops} \\
\m  \mathrm{FFTRate(SKA1Survey)}={{4.72 \times 10^{+16}\,\log \left({{1.77 \times 10^{+8}}\over{N_{\rm facet}^2}}\right)}\over{\mathrm{twsnap}}}\;\mathrm{Ops} \\
\m  \mbox{{}FFTRate(SKA1Low,[Nfacet = 1,twsnap = 40 ` s]){}}=4.5 \times 10^{+14}\;{{\mathrm{Ops}}\over{\mathrm{s}}} \\
\m  \mbox{{}FFTRate(SKA1Mid,[Nfacet = 1,twsnap = 200 ` s]){}}=2.28 \times 10^{+15}\;{{\mathrm{Ops}}\over{\mathrm{s}}} \\
\m  \mbox{{}FFTRate(SKA1Survey,[Nfacet = 1,twsnap = 550 ` s]){}}=1.63 \times 10^{+15}\;{{\mathrm{Ops}}\over{\mathrm{s}}} \\
\end{maxima}

\subsection{Reprojection rate}

In the snapshot technique each snapshot has to be re-projected to a
common coordinate system. If it is assumed that 50 FLOP\footnote{The
  assumption of 50 FLOP is a guess for the time being. Investigation
  of ASKAPSoft have shown that that greatest fraction of the
  reprojection cost is in computing the coordinate transformation
  rather than the interpolation.} are required per pixel to reproject
we can get following required rate:

\begin{maxima}[]
ReProjRate : (50 `Ops ) * NPix**2 * Nf * Nbeam * 4 * Nfacet**2 / twsnap;

calcTel(ReProjRate, SKA1Low);
calcTel(ReProjRate, SKA1Mid);
calcTel(ReProjRate, SKA1Survey);


\maximaoutput*
\m  {{3200\,B_{\rm max}^2\,N_{\rm beam}\,N_{\rm f}}\over{D_{\rm s}^2\,\mathrm{twsnap}}}\;\mathrm{Ops} \\
\m  \mathrm{ReProjRate(SKA1Low)}={{6.68 \times 10^{+15}}\over{\mathrm{twsnap}}}\;\mathrm{Ops} \\
\m  \mathrm{ReProjRate(SKA1Mid)}={{1.45 \times 10^{+17}}\over{\mathrm{twsnap}}}\;\mathrm{Ops} \\
\m  \mathrm{ReProjRate(SKA1Survey)}={{3.27 \times 10^{+17}}\over{\mathrm{twsnap}}}\;\mathrm{Ops} \\
\end{maxima}

\subsection{Gridding rate}


Gridding rate, expressed as function of snaphsot duration
\begin{maxima}[]
GridRate   : VisInRate * (NAA**2+'qty(NGW**2)) * (8 `Ops) * Nfacet**2;

calcTel(GridRate, SKA1Low);
calcTel(GridRate, SKA1Mid);
calcTel(GridRate, SKA1Survey);

\maximaoutput*
\m  {{16\,\left(N_{\rm a}-1\right)\,N_{\rm a}\,N_{\rm beam}\,N_{\rm f}\,N_{\rm facet}^2\,\left(\mathrm{qty}\left({{16\,\mathrm{qty}\left({{3.48 \times 10^{-6}\,B_{\rm max}\,\mathrm{twsnap}\,\sqrt{{{B_{\rm max}\,\mathrm{twsnap}}\over{\lambda}}}}\over{D_{\rm s}}}\;{{1}\over{\mathrm{s}^{{{3}\over{2}}}}}+{{3.3 \times 10^{-10}\,B_{\rm max}^2\,\mathrm{twsnap}^2}\over{D_{\rm s}^2}}\;{{1}\over{\mathrm{s}^2}}\right)\,\lambda^2}\over{D_{\rm s}^2}}\right)+N_{\rm AA}^2\right)}\over{t_{\rm dump}}}\;\mathrm{Ops} \\
\m  \mathrm{GridRate(SKA1Low)}=7.15 \times 10^{+12}\,N_{\rm facet}^2\,\left(0.1\,\left(0.002\,\mathrm{twsnap}^2+1.8\,\mathrm{twsnap}^{{{3}\over{2}}}\right)+81.\right)\;{{\mathrm{Ops}}\over{\mathrm{s}}} \\
\m  \mathrm{GridRate(SKA1Mid)}=3.29 \times 10^{+12}\,N_{\rm facet}^2\,\left(0.004\,\left(0.05\,\mathrm{twsnap}^2+40.\,\mathrm{twsnap}^{{{3}\over{2}}}\right)+81.\right)\;{{\mathrm{Ops}}\over{\mathrm{s}}} \\
\m  \mathrm{GridRate(SKA1Survey)}=4.48 \times 10^{+12}\,N_{\rm facet}^2\,\left(0.004\,\left(0.003\,\mathrm{twsnap}^2+5.1\,\mathrm{twsnap}^{{{3}\over{2}}}\right)+81.\right)\;{{\mathrm{Ops}}\over{\mathrm{s}}} \\
\end{maxima}

\subsection{Phase Rotation -- preliminary}

In the case of multi-facet imaging, it is necessary to phase rotate
visibilities to each new imaging plane. A first estimate of this can
be made as:

\begin{maxima}[]
PhaseRotateRate   : VisInRate * (20 `Ops) * Nfacet**2;
\maximaoutput*
\m  {{40\,\left(N_{\rm a}-1\right)\,N_{\rm a}\,N_{\rm beam}\,N_{\rm f}\,N_{\rm facet}^2}\over{t_{\rm dump}}}\;\mathrm{Ops} \\
\end{maxima}

Allowing for 20 operations to compute the sine/cosine and do the
multiplication.

\begin{maxima}[]
ev(PhaseRotateRate, append([Nfacet=2 * Bmax/ Ds], SKA1Low), infeval);
ev(PhaseRotateRate, append([Nfacet=2 * Bmax/ Ds], SKA1Mid), infeval);
\maximaoutput*
\m  5.83 \times 10^{+20}\;{{\mathrm{Ops}}\over{\mathrm{s}}} \\
\m  5.84 \times 10^{+21}\;{{\mathrm{Ops}}\over{\mathrm{s}}} \\
\end{maxima}

\subsection{Computing requirement}

Total compute rate is then given by:
\begin{maxima}[]
CompRate : FFTRate+ ReProjRate+ GridRate;

calcTel(CompRate, SKA1Low);
calcTel(CompRate, SKA1Mid);
calcTel(CompRate, SKA1Survey);
\maximaoutput*
\m  \left({{16\,\left(N_{\rm a}-1\right)\,N_{\rm a}\,N_{\rm beam}\,N_{\rm f}\,N_{\rm facet}^2\,\left(\mathrm{qty}\left({{16\,\mathrm{qty}\left({{3.48 \times 10^{-6}\,B_{\rm max}\,\mathrm{twsnap}\,\sqrt{{{B_{\rm max}\,\mathrm{twsnap}}\over{\lambda}}}}\over{D_{\rm s}}}\;{{1}\over{\mathrm{s}^{{{3}\over{2}}}}}+{{3.3 \times 10^{-10}\,B_{\rm max}^2\,\mathrm{twsnap}^2}\over{D_{\rm s}^2}}\;{{1}\over{\mathrm{s}^2}}\right)\,\lambda^2}\over{D_{\rm s}^2}}\right)+N_{\rm AA}^2\right)}\over{t_{\rm dump}}}+{{320\,B_{\rm max}^2\,N_{\rm beam}\,N_{\rm f}\,\log \left({{16\,B_{\rm max}^2}\over{D_{\rm s}^2\,N_{\rm facet}^2}}\right)}\over{\log 2\,D_{\rm s}^2\,\mathrm{twsnap}}}+{{3200\,B_{\rm max}^2\,N_{\rm beam}\,N_{\rm f}}\over{D_{\rm s}^2\,\mathrm{twsnap}}}\right)\;\mathrm{Ops} \\
\m  \mathrm{CompRate(SKA1Low)}=\left(7.15 \times 10^{+12}\,N_{\rm facet}^2\,\left(0.1\,\left(0.002\,\mathrm{twsnap}^2+1.8\,\mathrm{twsnap}^{{{3}\over{2}}}\right)+81.\right)\;{{1}\over{\mathrm{s}}}+{{9.64 \times 10^{+14}\,\log \left({{1.3 \times 10^{+8}}\over{N_{\rm facet}^2}}\right)}\over{\mathrm{twsnap}}}+{{6.68 \times 10^{+15}}\over{\mathrm{twsnap}}}\right)\;\mathrm{Ops} \\
\m  \mathrm{CompRate(SKA1Mid)}=\left(3.29 \times 10^{+12}\,N_{\rm facet}^2\,\left(0.004\,\left(0.05\,\mathrm{twsnap}^2+40.\,\mathrm{twsnap}^{{{3}\over{2}}}\right)+81.\right)\;{{1}\over{\mathrm{s}}}+{{2.1 \times 10^{+16}\,\log \left({{2.84 \times 10^{+9}}\over{N_{\rm facet}^2}}\right)}\over{\mathrm{twsnap}}}+{{1.45 \times 10^{+17}}\over{\mathrm{twsnap}}}\right)\;\mathrm{Ops} \\
\m  \mathrm{CompRate(SKA1Survey)}=\left(4.48 \times 10^{+12}\,N_{\rm facet}^2\,\left(0.004\,\left(0.003\,\mathrm{twsnap}^2+5.1\,\mathrm{twsnap}^{{{3}\over{2}}}\right)+81.\right)\;{{1}\over{\mathrm{s}}}+{{4.72 \times 10^{+16}\,\log \left({{1.77 \times 10^{+8}}\over{N_{\rm facet}^2}}\right)}\over{\mathrm{twsnap}}}+{{3.27 \times 10^{+17}}\over{\mathrm{twsnap}}}\right)\;\mathrm{Ops} \\
\end{maxima}
This is the compute rate per half-major cycle (i.e., per forward step
or backward step).


\subsection{Optimal w-snapshot duration time}

We can find the optimal w-snapshot duration time by minimising the
total compute rate with respect to this variable:
\begin{maxima}[]
find_root(diff(ev(CompRate, twsnap = xx `s, SKA1Low, Nfacet=1, nouns) / ( 1` (Ops/s)), xx), xx, 1, 10000);
find_root(diff(ev(CompRate, twsnap = xx `s, SKA1Mid, Nfacet=1, nouns) / ( 1` (Ops/s)), xx), xx, 1, 10000);
find_root(diff(ev(CompRate, twsnap = xx `s, SKA1Survey, Nfacet=1, nouns) / ( 1` (Ops/s)), xx), xx, 1, 10000);

\maximaoutput*
\m  40. \\
\m  200. \\
\m  550. \\
\end{maxima}


\subsection{Compute rate at optimal w-snapshot time}
\label{sec:wsnapshot-opt-rate}

If we take the optimised snapshot times from above we get:

\begin{maxima}[]
calcTelA(CompRate, SKA1Low, [Nfacet=1, twsnap=40`s]);
calcTelA(CompRate, SKA1Mid, [Nfacet=1, twsnap=200`s]);
calcTelA(CompRate, SKA1Survey, [Nfacet=1, twsnap=550`s]);

\maximaoutput*
\m  \mbox{{}CompRate(SKA1Low,[Nfacet = 1,twsnap = 40 ` s]){}}=1.58 \times 10^{+15}\;{{\mathrm{Ops}}\over{\mathrm{s}}} \\
\m  \mbox{{}CompRate(SKA1Mid,[Nfacet = 1,twsnap = 200 ` s]){}}=5.12 \times 10^{+15}\;{{\mathrm{Ops}}\over{\mathrm{s}}} \\
\m  \mbox{{}CompRate(SKA1Survey,[Nfacet = 1,twsnap = 550 ` s]){}}=4.02 \times 10^{+15}\;{{\mathrm{Ops}}\over{\mathrm{s}}} \\
\end{maxima}
again this is per half-major cycle.

One thing to note is that kernel size of $\sqrt(9^2+7^2)\sim 11$ will
likely not be enough to capture the A term for SKA1 Low. It may
therefore be necessary to use kernels which are larger than this for
SKA1 Low, leading to higher computational load. 

\section{Working memory and computational intensity}

\subsection{Computational intensity}

A crude first estimate of the computational intensity can be estimated
as the ratio of the largest data set (the visibilities) and the
processing computing requirements computed estimated above.

Note that: 
\begin{itemize}
\item Here I assume 8 bytes per visibility which assumes a very
  efficient data format, a conservative estimate would be at least 16
  bytes
\item The division is of the visibility rate vs half-cycle compute
  requirements. This number is therefore independent of the number of
  cycles
\end{itemize}

\begin{maxima}[]
CompIntens1: CompRate /  (VisInRate * 8 `byte);

calcTelA(CompIntens1, SKA1Low, [Nfacet=1, twsnap=40`s] );
calcTelA(CompIntens1, SKA1Mid, [Nfacet=1, twsnap=200`s]);
calcTelA(CompIntens1, SKA1Survey, [Nfacet=1, twsnap=550`s]);

\maximaoutput*
\m  {{\left(N_{\rm a}-1\right)^ {- 1 }\,N_{\rm a}^ {- 1 }\,N_{\rm beam}^ {- 1 }\,N_{\rm f}^ {- 1 }\,t_{\rm dump}\,\left({{16\,\left(N_{\rm a}-1\right)\,N_{\rm a}\,N_{\rm beam}\,N_{\rm f}\,N_{\rm facet}^2\,\left(\mathrm{qty}\left({{16\,\mathrm{qty}\left({{3.48 \times 10^{-6}\,B_{\rm max}\,\mathrm{twsnap}\,\sqrt{{{B_{\rm max}\,\mathrm{twsnap}}\over{\lambda}}}}\over{D_{\rm s}}}\;{{1}\over{\mathrm{s}^{{{3}\over{2}}}}}+{{3.3 \times 10^{-10}\,B_{\rm max}^2\,\mathrm{twsnap}^2}\over{D_{\rm s}^2}}\;{{1}\over{\mathrm{s}^2}}\right)\,\lambda^2}\over{D_{\rm s}^2}}\right)+N_{\rm AA}^2\right)}\over{t_{\rm dump}}}+{{320\,B_{\rm max}^2\,N_{\rm beam}\,N_{\rm f}\,\log \left({{16\,B_{\rm max}^2}\over{D_{\rm s}^2\,N_{\rm facet}^2}}\right)}\over{\log 2\,D_{\rm s}^2\,\mathrm{twsnap}}}+{{3200\,B_{\rm max}^2\,N_{\rm beam}\,N_{\rm f}}\over{D_{\rm s}^2\,\mathrm{twsnap}}}\right)}\over{16}}\;{{\mathrm{Ops}}\over{\mathrm{byte}}} \\
\m  \mbox{{}CompIntens1(SKA1Low,[Nfacet = 1,twsnap = 40 ` s]){}}=220.\;{{\mathrm{Ops}}\over{\mathrm{byte}}} \\
\m  \mbox{{}CompIntens1(SKA1Mid,[Nfacet = 1,twsnap = 200 ` s]){}}=1500.\;{{\mathrm{Ops}}\over{\mathrm{byte}}} \\
\m  \mbox{{}CompIntens1(SKA1Survey,[Nfacet = 1,twsnap = 550 ` s]){}}=890.\;{{\mathrm{Ops}}\over{\mathrm{byte}}} \\
\end{maxima}

Note that if faceting is done in the simplest way, i.e., if the entire
visibility set is iterated for each facet from the beginning, then the
computational intensity \emph{decreases} in rough proportion to the
number of facets. The logical approach would therefore be phase rotate
to each facet, average the data in time, and record the averaged data. 

\subsection{Working memory requirements}

The working random access memory required to process each channel
comprises of the target grid in $uv$ plane and potentially the cached
convolution functions. The target grid size is double precision
complex number with a double precision weight; therefore:

\begin{maxima}[]
TargetGridSize: NPix**2 * (16+8)`byte;

ev(TargetGridSize, append([Nfacet=1], SKA1Low)) `` Gbyte;
ev(TargetGridSize, append([Nfacet=1], SKA1Mid)) `` Gbyte;
ev(TargetGridSize, append([Nfacet=1], SKA1Survey)) `` Gbyte;
\maximaoutput*
\m  {{384\,B_{\rm max}^2}\over{D_{\rm s}^2\,N_{\rm facet}^2}}\;\mathrm{byte} \\
\m  3.1\;\mathrm{Gbyte} \\
\m  68.\;\mathrm{Gbyte} \\
\m  4.2\;\mathrm{Gbyte} \\
\end{maxima}

Keeping all of the target grids in memory for the most challenging
case would require (factor of four to account for four polarisation
products):
\begin{maxima}[]
ev(TargetGridSize* Nf *4 * Nbeam  , append([Nfacet=1], SKA1Low)) `` Pbyte;
ev(TargetGridSize*Nf * 4 * Nbeam , append([Nfacet=1], SKA1Mid)) `` Pbyte;
ev(TargetGridSize*Nf * 4 * Nbeam , append([Nfacet=1], SKA1Survey)) `` Pbyte;

\maximaoutput*
\m  3.2\;\mathrm{Pbyte} \\
\m  69.\;\mathrm{Pbyte} \\
\m  150.\;\mathrm{Pbyte} \\
\end{maxima}
of random access memory \footnote{Current top supercomputer has 1
  PetaByte of RAM}. This requirement can however be reduced both
sequentially processing the channels rather than keeping all of them
going at the same time or by employing faceting.

\subsection{Computation intensity in respect of working memory}

\subsubsection{Keeping all target grids in memory}

If we wanted to keep all target grids in memory we would need the
following balance between ops and memory:
\begin{maxima}[]

FiWorkingMem: ( (CompRate * 20)  ) / (TargetGridSize* Nf *4 * Nbeam);

float(ev(FiWorkingMem    , append([Nfacet=1, NGW=6], SKA1Low)
,infeval)) `` ((Ops/s)/Gbyte  );
float(ev(FiWorkingMem , append([Nfacet=1, NGW=35], SKA1Mid) ,infeval))`` ((Ops/s)/Gbyte  );
float(ev(FiWorkingMem, append([Nfacet=1, NGW=25], SKA1Survey)
,infeval)) `` ((Ops/s)/Gbyte  );

\maximaoutput*
\m  {{5\,B_{\rm max}^ {- 2 }\,D_{\rm s}^2\,N_{\rm beam}^ {- 1 }\,N_{\rm f}^ {- 1 }\,N_{\rm facet}^2\,\left({{16\,\left(N_{\rm a}-1\right)\,N_{\rm a}\,N_{\rm beam}\,N_{\rm f}\,N_{\rm facet}^2\,\left(\mathrm{qty}\left({{16\,\mathrm{qty}\left({{3.48 \times 10^{-6}\,B_{\rm max}\,\mathrm{twsnap}\,\sqrt{{{B_{\rm max}\,\mathrm{twsnap}}\over{\lambda}}}}\over{D_{\rm s}}}\;{{1}\over{\mathrm{s}^{{{3}\over{2}}}}}+{{3.3 \times 10^{-10}\,B_{\rm max}^2\,\mathrm{twsnap}^2}\over{D_{\rm s}^2}}\;{{1}\over{\mathrm{s}^2}}\right)\,\lambda^2}\over{D_{\rm s}^2}}\right)+N_{\rm AA}^2\right)}\over{t_{\rm dump}}}+{{320\,B_{\rm max}^2\,N_{\rm beam}\,N_{\rm f}\,\log \left({{16\,B_{\rm max}^2}\over{D_{\rm s}^2\,N_{\rm facet}^2}}\right)}\over{\log 2\,D_{\rm s}^2\,\mathrm{twsnap}}}+{{3200\,B_{\rm max}^2\,N_{\rm beam}\,N_{\rm f}}\over{D_{\rm s}^2\,\mathrm{twsnap}}}\right)}\over{384}}\;{{\mathrm{Ops}}\over{\mathrm{byte}}} \\
\m  6.23 \times 10^{-6}\,s\,\left(7.15 \times 10^{+12}\,\left(\mathrm{qty}\left(0.1\,\mathrm{qty}\left(0.002\,\mathrm{twsnap}^2\;{{1}\over{\mathrm{s}^2}}+1.8\,\mathrm{twsnap}^{{{3}\over{2}}}\;{{\sqrt{\mathrm{Hz}}}\over{\mathrm{s}}}\right)\;{{1}\over{\mathrm{s}^2\,\mathrm{Hz}^2}}\right)+81.\right)\;{{1}\over{\mathrm{s}}}+{{2.47 \times 10^{+16}}\over{\mathrm{twsnap}}}\right)\;{{\mathrm{Ops}}\over{\mathrm{s}\,\mathrm{Gbyte}}} \\
\m  2.86 \times 10^{-7}\,s\,\left(3.29 \times 10^{+12}\,\left(\mathrm{qty}\left(0.004\,\mathrm{qty}\left(0.05\,\mathrm{twsnap}^2\;{{1}\over{\mathrm{s}^2}}+40.\,\mathrm{twsnap}^{{{3}\over{2}}}\;{{\sqrt{\mathrm{Hz}}}\over{\mathrm{s}}}\right)\;{{1}\over{\mathrm{s}^2\,\mathrm{Hz}^2}}\right)+81.\right)\;{{1}\over{\mathrm{s}}}+{{6.03 \times 10^{+17}}\over{\mathrm{twsnap}}}\right)\;{{\mathrm{Ops}}\over{\mathrm{s}\,\mathrm{Gbyte}}} \\
\m  1.27 \times 10^{-7}\,s\,\left(4.48 \times 10^{+12}\,\left(\mathrm{qty}\left(0.004\,\mathrm{qty}\left(0.003\,\mathrm{twsnap}^2\;{{1}\over{\mathrm{s}^2}}+5.1\,\mathrm{twsnap}^{{{3}\over{2}}}\;{{\sqrt{\mathrm{Hz}}}\over{\mathrm{s}}}\right)\;{{1}\over{\mathrm{s}^2\,\mathrm{Hz}^2}}\right)+81.\right)\;{{1}\over{\mathrm{s}}}+{{1.22 \times 10^{+18}}\over{\mathrm{twsnap}}}\right)\;{{\mathrm{Ops}}\over{\mathrm{s}\,\mathrm{Gbyte}}} \\
\end{maxima}
This is lower than the expected ratio of typical machines at
deployment time. The assumption therefore should be that not all grids
are kept in memory. 

\subsubsection{Shuffling target grids: I/O intensity }

If we assume target grids are shuffled out of main memory each
snapshot once to account for sequential processing, then the I/O
intensity is:
\begin{maxima}[]

CiWorkingMem: ( (CompRate * 20)  * twsnap) / (TargetGridSize* Nf *4 * Nbeam);

calcTelA(CiWorkingMem, SKA1Low, [Nfacet=1, twsnap=40`s] );
calcTelA(CiWorkingMem, SKA1Mid, [Nfacet=1, twsnap=200`s]);
calcTelA(CiWorkingMem, SKA1Survey, [Nfacet=1, twsnap=550`s]);


\maximaoutput*
\m  {{5\,B_{\rm max}^ {- 2 }\,D_{\rm s}^2\,N_{\rm beam}^ {- 1 }\,N_{\rm f}^ {- 1 }\,N_{\rm facet}^2\,\mathrm{twsnap}\,\left({{16\,\left(N_{\rm a}-1\right)\,N_{\rm a}\,N_{\rm beam}\,N_{\rm f}\,N_{\rm facet}^2\,\left(\mathrm{qty}\left({{16\,\mathrm{qty}\left({{3.48 \times 10^{-6}\,B_{\rm max}\,\mathrm{twsnap}\,\sqrt{{{B_{\rm max}\,\mathrm{twsnap}}\over{\lambda}}}}\over{D_{\rm s}}}\;{{1}\over{\mathrm{s}^{{{3}\over{2}}}}}+{{3.3 \times 10^{-10}\,B_{\rm max}^2\,\mathrm{twsnap}^2}\over{D_{\rm s}^2}}\;{{1}\over{\mathrm{s}^2}}\right)\,\lambda^2}\over{D_{\rm s}^2}}\right)+N_{\rm AA}^2\right)}\over{t_{\rm dump}}}+{{320\,B_{\rm max}^2\,N_{\rm beam}\,N_{\rm f}\,\log \left({{16\,B_{\rm max}^2}\over{D_{\rm s}^2\,N_{\rm facet}^2}}\right)}\over{\log 2\,D_{\rm s}^2\,\mathrm{twsnap}}}+{{3200\,B_{\rm max}^2\,N_{\rm beam}\,N_{\rm f}}\over{D_{\rm s}^2\,\mathrm{twsnap}}}\right)}\over{384}}\;{{\mathrm{Ops}}\over{\mathrm{byte}}} \\
\m  \mbox{{}CiWorkingMem(SKA1Low,[Nfacet = 1,twsnap = 40 ` s]){}}=390.\;{{\mathrm{Ops}}\over{\mathrm{byte}}} \\
\m  \mbox{{}CiWorkingMem(SKA1Mid,[Nfacet = 1,twsnap = 200 ` s]){}}=290.\;{{\mathrm{Ops}}\over{\mathrm{byte}}} \\
\m  \mbox{{}CiWorkingMem(SKA1Survey,[Nfacet = 1,twsnap = 550 ` s]){}}=280.\;{{\mathrm{Ops}}\over{\mathrm{byte}}} \\
\end{maxima}

These are relatively small numbers. This amount of shuffling can be
avoided by processing all snapshots of a single frequency channel in
sequence and accumulating the result. In this case these numbers are
increased by the number of snapshots in observation (usually 60 or
more).


\subsection{Working memory limited machine}

Say we have 1PByte of working memory for target grids. We can do
following number of target grids and fraction of total
\begin{maxima}[]
TargetMemBudget: (1 `Pbyte);

NGrids : TargetMemBudget/ TargetGridSize;

ev(NGrids, append([Nfacet=1], SKA1Low)) `` 1;
ev(NGrids, append([Nfacet=1], SKA1Mid))  `` 1;
ev(NGrids, append([Nfacet=1], SKA1Survey)) ``1;

ev((Nf*4*Nbeam)/  (TargetMemBudget/ TargetGridSize), append([Nfacet=1], SKA1Low)) `` 1;
ev((Nf*4*Nbeam)/ (TargetMemBudget/TargetGridSize), append([Nfacet=1], SKA1Mid))  `` 1;
ev((Nf*4*Nbeam)/ (TargetMemBudget/TargetGridSize), append([Nfacet=1], SKA1Survey)) ``1;


\maximaoutput*
\m  1\;\mathrm{Pbyte} \\
\m  {{B_{\rm max}^ {- 2 }\,D_{\rm s}^2\,N_{\rm facet}^2}\over{384}}\;{{\mathrm{Pbyte}}\over{\mathrm{byte}}} \\
\m  310000. \\
\m  14000. \\
\m  230000. \\
\m  3.2 \\
\m  69. \\
\m  150. \\
\end{maxima}

We can then calculate the number of FLOPs that need to work on each
target grid:

\begin{maxima}[]

FlopsPerGrid : ((CompRate*20)/NGrids)``1;

calcTelA(FlopsPerGrid, SKA1Low, [Nfacet=1, twsnap=40`s] );
calcTelA(FlopsPerGrid, SKA1Mid, [Nfacet=1, twsnap=200`s]);
calcTelA(FlopsPerGrid, SKA1Survey, [Nfacet=1, twsnap=550`s]);

\maximaoutput*
\m  {{3\,B_{\rm max}^2\,D_{\rm s}^ {- 2 }\,N_{\rm facet}^ {- 2 }\,\mathrm{Ops}\,\left({{16\,\left(N_{\rm a}-1\right)\,N_{\rm a}\,N_{\rm beam}\,N_{\rm f}\,N_{\rm facet}^2\,\left(\mathrm{qty}\left({{16\,\mathrm{qty}\left({{3.48 \times 10^{-6}\,B_{\rm max}\,\mathrm{twsnap}\,\sqrt{{{B_{\rm max}\,\mathrm{twsnap}}\over{\lambda}}}}\over{D_{\rm s}}}\;{{1}\over{\mathrm{s}^{{{3}\over{2}}}}}+{{3.3 \times 10^{-10}\,B_{\rm max}^2\,\mathrm{twsnap}^2}\over{D_{\rm s}^2}}\;{{1}\over{\mathrm{s}^2}}\right)\,\lambda^2}\over{D_{\rm s}^2}}\right)+N_{\rm AA}^2\right)}\over{t_{\rm dump}}}+{{320\,B_{\rm max}^2\,N_{\rm beam}\,N_{\rm f}\,\log \left({{16\,B_{\rm max}^2}\over{D_{\rm s}^2\,N_{\rm facet}^2}}\right)}\over{\log 2\,D_{\rm s}^2\,\mathrm{twsnap}}}+{{3200\,B_{\rm max}^2\,N_{\rm beam}\,N_{\rm f}}\over{D_{\rm s}^2\,\mathrm{twsnap}}}\right)}\over{390625000000}} \\
\m  \mbox{{}FlopsPerGrid(SKA1Low,[Nfacet = 1,twsnap = 40 ` s]){}}=9.95 \times 10^{+10}\,\mathrm{Ops}\;{{1}\over{\mathrm{s}}} \\
\m  \mbox{{}FlopsPerGrid(SKA1Mid,[Nfacet = 1,twsnap = 200 ` s]){}}=7.0 \times 10^{+12}\,\mathrm{Ops}\;{{1}\over{\mathrm{s}}} \\
\m  \mbox{{}FlopsPerGrid(SKA1Survey,[Nfacet = 1,twsnap = 550 ` s]){}}=3.43 \times 10^{+11}\,\mathrm{Ops}\;{{1}\over{\mathrm{s}}} \\
\end{maxima}

\section{Computing the convolution kernels}

Computing the convolution functions is a fairly complex process which
in detail depends on the effects that are corrected for using the
convolution function in the $uv$-plane. However, the major
computational complexity arises if the convolution functions need to
be computed on a baseline-by-baseline basis to account for the
different $A$ responses of the two receiving elements forming the
baseline.  In this case the major computational cost is the FFT
between the image plane where combination of effects is done by
multiplication.  The computational cost can therefore be estimated as:

\begin{maxima}[]
gcfc1 : [NGCF = sqrt(NAA**2+'qty(NGW**2)),
         FNGCF = Qgcf * NGCF] ;
gcfrate : (5 `Ops) *  (FNGCF ** 2) * log2(FNGCF ** 2)  / Tion * Nbl *
Qfcv / 10 *
Nfacet**2;


\maximaoutput*
\m  \left[ \mathrm{NGCF}=\sqrt{\mathrm{qty}\left({{16\,\mathrm{qty}\left({{3.48 \times 10^{-6}\,B_{\rm max}\,\mathrm{twsnap}\,\sqrt{{{B_{\rm max}\,\mathrm{twsnap}}\over{\lambda}}}}\over{D_{\rm s}}}\;{{1}\over{\mathrm{s}^{{{3}\over{2}}}}}+{{3.3 \times 10^{-10}\,B_{\rm max}^2\,\mathrm{twsnap}^2}\over{D_{\rm s}^2}}\;{{1}\over{\mathrm{s}^2}}\right)\,\lambda^2}\over{D_{\rm s}^2}}\right)+N_{\rm AA}^2} , \mathrm{FNGCF}=\mathrm{Qgcf}\,\mathrm{NGCF} \right] \\
\m  {{2\,\left(N_{\rm a}-1\right)\,N_{\rm a}\,N_{\rm beam}\,N_{\rm f}\,N_{\rm facet}^2\,\mathrm{Qfcv}\,\mathrm{FNGCF}^2\,\log \mathrm{FNGCF}}\over{\log 2\,t_{\rm ionsph}}}\;\mathrm{Ops} \\
\end{maxima}

The ratio of operation count for imaging (FFT+Reprojection+Gridding)
to operation count for the convolution kernel can be expressed as
function of the ionospheric timescale:

\begin{maxima}[]

ConvToGridRatio : ev(gcfrate/ GridRate, gcfc1);

calcTelA(ConvToGridRatio, SKA1Low, [Nfacet=1, twsnap=40`s, Qfcv=1, Qgcf=8, gcfc1] );
calcTelA(ConvToGridRatio, SKA1Mid, [Nfacet=1, twsnap=200`s, Qfcv=1, Qgcf=8, gcfc1]);
calcTelA(ConvToGridRatio, SKA1Survey, [Nfacet=1, twsnap=550`s, Qfcv=1, Qgcf=8, gcfc1]);

\maximaoutput*
\m  {{\mathrm{Qfcv}\,\mathrm{Qgcf}^2\,t_{\rm dump}\,\mathrm{NGCF}^2\,\log \left(\mathrm{Qgcf}\,\mathrm{NGCF}\right)\,\left(\mathrm{qty}\left({{16\,\mathrm{qty}\left({{3.48 \times 10^{-6}\,B_{\rm max}\,\mathrm{twsnap}\,\sqrt{{{B_{\rm max}\,\mathrm{twsnap}}\over{\lambda}}}}\over{D_{\rm s}}}\;{{1}\over{\mathrm{s}^{{{3}\over{2}}}}}+{{3.3 \times 10^{-10}\,B_{\rm max}^2\,\mathrm{twsnap}^2}\over{D_{\rm s}^2}}\;{{1}\over{\mathrm{s}^2}}\right)\,\lambda^2}\over{D_{\rm s}^2}}\right)+N_{\rm AA}^2\right)^ {- 1 }}\over{8\,\log 2\,t_{\rm ionsph}}} \\
\m  \mbox{{}ConvToGridRatio(SKA1Low,[Nfacet = 1,twsnap = 40 ` s,Qfcv = 1,Qgcf = 8,gcfc1]){}}={{31.}\over{t_{\rm ionsph}}}\;\mathrm{s} \\
\m  \mbox{{}ConvToGridRatio(SKA1Mid,[Nfacet = 1,twsnap = 200 ` s,Qfcv = 1,Qgcf = 8,gcfc1]){}}={{4.9}\over{t_{\rm ionsph}}}\;\mathrm{s} \\
\m  \mbox{{}ConvToGridRatio(SKA1Survey,[Nfacet = 1,twsnap = 550 ` s,Qfcv = 1,Qgcf = 8,gcfc1]){}}={{17.}\over{t_{\rm ionsph}}}\;\mathrm{s} \\
\end{maxima}

The operation rate as function  of ionospheric timescale is given by:
\begin{maxima}[]
calcTelA(gcfrate, SKA1Low, [Nfacet=1, twsnap=40`s, Qfcv=1, Qgcf=8, gcfc1] );
calcTelA(gcfrate, SKA1Mid, [Nfacet=1, twsnap=200`s, Qfcv=1, Qgcf=8, gcfc1]);
calcTelA(gcfrate, SKA1Survey, [Nfacet=1, twsnap=550`s, Qfcv=1, Qgcf=8, gcfc1]);
\maximaoutput*
\m  \mbox{{}gcfrate(SKA1Low,[Nfacet = 1,twsnap = 40 ` s,Qfcv = 1,Qgcf = 8,gcfc1]){}}={{3.04 \times 10^{+16}}\over{t_{\rm ionsph}}}\;\mathrm{Ops} \\
\m  \mbox{{}gcfrate(SKA1Mid,[Nfacet = 1,twsnap = 200 ` s,Qfcv = 1,Qgcf = 8,gcfc1]){}}={{1.03 \times 10^{+16}}\over{t_{\rm ionsph}}}\;\mathrm{Ops} \\
\m  \mbox{{}gcfrate(SKA1Survey,[Nfacet = 1,twsnap = 550 ` s,Qfcv = 1,Qgcf = 8,gcfc1]){}}={{3.14 \times 10^{+16}}\over{t_{\rm ionsph}}}\;\mathrm{Ops} \\
\end{maxima}

If we assume 60 second ionospheric time rate, we get following
operation rates (per half-major-cycle as everywhere).
\begin{maxima}[]
calcTelA(gcfrate, SKA1Low, [Nfacet=1, twsnap=40`s, Qfcv=1, Qgcf=8,  gcfc1, Tion=60`s] );
calcTelA(gcfrate, SKA1Mid, [Nfacet=1, twsnap=200`s, Qfcv=1, Qgcf=8, gcfc1, Tion=60`s]);
calcTelA(gcfrate, SKA1Survey, [Nfacet=1, twsnap=550`s, Qfcv=1, Qgcf=8, gcfc1, Tion=60`s]);

\maximaoutput*
\m  \mbox{{}gcfrate(SKA1Low,[Nfacet = 1,twsnap = 40 ` s,Qfcv = 1,Qgcf = 8,gcfc1,Tion = 60 ` s]){}}=5.07 \times 10^{+14}\;{{\mathrm{Ops}}\over{\mathrm{s}}} \\
\m  \mbox{{}gcfrate(SKA1Mid,[Nfacet = 1,twsnap = 200 ` s,Qfcv = 1,Qgcf = 8,gcfc1,Tion = 60 ` s]){}}=1.72 \times 10^{+14}\;{{\mathrm{Ops}}\over{\mathrm{s}}} \\
\m  \mbox{{}gcfrate(SKA1Survey,[Nfacet = 1,twsnap = 550 ` s,Qfcv = 1,Qgcf = 8,gcfc1,Tion = 60 ` s]){}}=5.24 \times 10^{+14}\;{{\mathrm{Ops}}\over{\mathrm{s}}} \\
\end{maxima}

\section{DFT source prediction}

Taking the estimate for the number of FLOPs for DFT by
\cite{Salvini2014Memo7} (one sided computation, not taking into
account aperture array-specifics) we can compute the total FLOP count
to predict the each visibility once for 10000 sources by the DFT
method:

\begin{maxima}[]
DFTRate: ((72` Ops)  * Na * (Na-1) * SDft + (256`Ops) * Na * SDft)* Nf * Nbeam / tdump;

calcTelA(DFTRate, SKA1Low, [SDft=1e4]);
calcTelA(DFTRate, SKA1Mid, [SDft=1e4]);
calcTelA(DFTRate, SKA1Survey, [SDft=1e4]);
\maximaoutput*
\m  {{N_{\rm beam}\,N_{\rm f}\,\left(72\,\left(N_{\rm a}-1\right)\,N_{\rm a}\,\mathrm{SDft}+256\,N_{\rm a}\,\mathrm{SDft}\right)}\over{t_{\rm dump}}}\;\mathrm{Ops} \\
\m  \mbox{{}DFTRate(SKA1Low,[SDft = 10000.]){}}=3.22 \times 10^{+17}\;{{\mathrm{Ops}}\over{\mathrm{s}}} \\
\m  \mbox{{}DFTRate(SKA1Mid,[SDft = 10000.]){}}=1.5 \times 10^{+17}\;{{\mathrm{Ops}}\over{\mathrm{s}}} \\
\m  \mbox{{}DFTRate(SKA1Survey,[SDft = 10000.]){}}=2.09 \times 10^{+17}\;{{\mathrm{Ops}}\over{\mathrm{s}}} \\
\end{maxima}



\section{Scenario: SKA1Mid with 120\,km Baselines}

In-prep ECP: SKA1Mid to have 120\,km maximum baseline and
proportionally longer dump time.


\begin{maxima}[]
SKA1MidECP : [ tdump = (0.08 * 200/120) `s, Na = 190+64 , Nf= 256000,
Nbeam=1, Ds=15`m, lambda=constvalue(%c) / (1160e6`Hz) , Bmax = 120e3`m, NAA=9];
\maximaoutput*
\m  \left[ t_{\rm dump}=0.1\;\mathrm{s} , N_{\rm a}=254 , N_{\rm f}=256000 , N_{\rm beam}=1 , D_{\rm s}=15\;\mathrm{m} , \lambda=0.2\;{{\mathrm{m}}\over{\mathrm{s}\,\mathrm{Hz}}} , B_{\rm max}=120000.\;\mathrm{m} , N_{\rm AA}=9 \right] \\
\end{maxima}

The optimal w-kernel/ snapshot duration changes to:

\begin{maxima}[]
  find_root(diff(ev(CompRate, twsnap = xx `s, SKA1MidECP, Nfacet=1, nouns) / ( 1` (Ops/s)), xx), xx, 1, 10000);
\maximaoutput*
\m  220. \\
\end{maxima}

At optimal w-snapshot duration the compute FLOP count for gridding,FFT
and coordinate projection according to the model is:
\begin{maxima}[]
calcTelA(CompRate, SKA1MidECP, [Nfacet=1, twsnap=220`s]);
\maximaoutput*
\m  \mbox{{}CompRate(SKA1MidECP,[Nfacet = 1,twsnap = 220 ` s]){}}=1.7 \times 10^{+15}\;{{\mathrm{Ops}}\over{\mathrm{s}}} \\
\end{maxima}


\section{Alternative to snapshots: W-stacking (to be reviesd)} 

For outline on ``w-stacking'' see also
\cite{VoronkovCalim2010Gridding}, and \cite{2013A&A...553A.105T}. The
concept of this technique is that visibilities with different values
of the $w$ coordinate are not combined in the $uv$ domain but rather
gridded separately and transformed into the image (maybe we should be
calling it the ``$lm$''?) domain, corrected for the $w$ term and then
combined. This obviously requires a degree of sorting of the data in
the $w$ dimension.

Compared to snapshots:
\begin{enumerate}
  \item The input visibility rate is the same. The cost of sorting the
    data is not considered in this section.
  \item The whole observation is processed together. The largest $w$
    that needs to be corrected is controlled by (half) the separation
    of the $w$ planes
  \item No re-projection like in $w$-snapshots is needed but instead a
    correction of the $w$ term is required for each plane after the
    FFT to image plane
  \item Number of FFTs is the same as number w planes
\end{enumerate}

\subsection{Computational load calculation}


\subsubsection{Size of convolution kernel}

The size of the required convolution is controlled by the largest
residual w-term after assigning each visibility to the nearest $w$
plane. To be consistent with previous sections we assume $w$-plane are
equidistant and the largest $w$ is simply half the distance between
the planes.  This gives the relationship between the size of kernel
and number of $w$-lanes.

\begin{maxima}[]
wsDeltaW : Bmax / 2 / wsNWPlanes;
wsNGW :  (Qw * wsDeltaW ) / Ds / Nfacet;

calcTel(wsNGW, SKA1Low);
calcTel(wsNGW, SKA1Mid);
calcTel(wsNGW, SKA1Survey);

\maximaoutput*
\m  {{B_{\rm max}}\over{2\,\mathrm{wsNWPlanes}}} \\
\m  {{B_{\rm max}}\over{2\,D_{\rm s}\,N_{\rm facet}\,\mathrm{wsNWPlanes}}} \\
\m  \mathrm{wsNGW(SKA1Low)}={{1400.}\over{N_{\rm facet}\,\mathrm{wsNWPlanes}}} \\
\m  \mathrm{wsNGW(SKA1Mid)}={{6600.}\over{N_{\rm facet}\,\mathrm{wsNWPlanes}}} \\
\m  \mathrm{wsNGW(SKA1Survey)}={{1600.}\over{N_{\rm facet}\,\mathrm{wsNWPlanes}}} \\
\end{maxima}

\subsubsection{Gridding}

The gridding rate is as before controlled simply by the size of the
convolution kernel and the input data rate, plus any faceting.

\begin{maxima}[]
wsGridRate   : VisInRate * (NAA**2+wsNGW**2) * (4 `Ops) * Nfacet**2;

calcTel(wsGridRate, SKA1Low);
calcTel(wsGridRate, SKA1Mid);
calcTel(wsGridRate, SKA1Survey);

\maximaoutput*
\m  {{8\,\left(N_{\rm a}-1\right)\,N_{\rm a}\,N_{\rm beam}\,N_{\rm f}\,N_{\rm facet}^2\,\left({{B_{\rm max}^2}\over{4\,D_{\rm s}^2\,N_{\rm facet}^2\,\mathrm{wsNWPlanes}^2}}+N_{\rm AA}^2\right)}\over{t_{\rm dump}}}\;\mathrm{Ops} \\
\m  \mathrm{wsGridRate(SKA1Low)}=3.57 \times 10^{+12}\,N_{\rm facet}^2\,\left({{2000000.}\over{N_{\rm facet}^2\,\mathrm{wsNWPlanes}^2}}+81.\right)\;{{\mathrm{Ops}}\over{\mathrm{s}}} \\
\m  \mathrm{wsGridRate(SKA1Mid)}=1.64 \times 10^{+12}\,N_{\rm facet}^2\,\left({{4.44 \times 10^{+7}}\over{N_{\rm facet}^2\,\mathrm{wsNWPlanes}^2}}+81.\right)\;{{\mathrm{Ops}}\over{\mathrm{s}}} \\
\m  \mathrm{wsGridRate(SKA1Survey)}=2.24 \times 10^{+12}\,N_{\rm facet}^2\,\left({{2700000.}\over{N_{\rm facet}^2\,\mathrm{wsNWPlanes}^2}}+81.\right)\;{{\mathrm{Ops}}\over{\mathrm{s}}} \\
\end{maxima}

\subsubsection{Image-plane correction}

The image plane correction is the multiplication by the term:
\begin{equation}
  \exp{\left[ i 2 \pi  w \left(1- \sqrt{1 - l^2 - m^2}\right) \right] }
\end{equation}
This requires computation one square root and one exponential. As a
very rough guess we assume this is about the same computational
complexity as the re-projection, i.e., 50 operations.

\begin{maxima}[]
wsImCorrectTot : (50 `Ops ) * NPix**2 * Nf * Nbeam * 4 * wsNWPlanes * Nfacet**2 ;

calcTel(wsImCorrectTot, SKA1Low);
calcTel(wsImCorrectTot, SKA1Mid);
calcTel(wsImCorrectTot, SKA1Survey);

\maximaoutput*
\m  {{3200\,B_{\rm max}^2\,N_{\rm beam}\,N_{\rm f}\,\mathrm{wsNWPlanes}}\over{D_{\rm s}^2}}\;\mathrm{Ops} \\
\m  \mathrm{wsImCorrectTot(SKA1Low)}=6.68 \times 10^{+15}\,\mathrm{wsNWPlanes}\;\mathrm{Ops} \\
\m  \mathrm{wsImCorrectTot(SKA1Mid)}=1.45 \times 10^{+17}\,\mathrm{wsNWPlanes}\;\mathrm{Ops} \\
\m  \mathrm{wsImCorrectTot(SKA1Survey)}=3.27 \times 10^{+17}\,\mathrm{wsNWPlanes}\;\mathrm{Ops} \\
\end{maxima}

\subsubsection{FFT compute requirement}

The number of FFTs is simply the number of $w$-planes times the usual
number of images and facets. Since the in $w$-stacking the whole
observation is processed together the calculation here is for the
total observation (and is not a rate).

\begin{maxima}[]
wsFFTTot: (5 `Ops) * NPix **2 * log2(NPix **2)  * Nf * Nbeam * 4 *
Nfacet**2 * wsNWPlanes;
\maximaoutput*
\m  {{320\,B_{\rm max}^2\,N_{\rm beam}\,N_{\rm f}\,\log \left({{16\,B_{\rm max}^2}\over{D_{\rm s}^2\,N_{\rm facet}^2}}\right)\,\mathrm{wsNWPlanes}}\over{\log 2\,D_{\rm s}^2}}\;\mathrm{Ops} \\
\end{maxima}

\subsubsection{Total compute requirement}

This is the computation of the total compute required (per half-cycle)
after optimising the number of $w$-planes. These numbers can be
compared directly to those in Section~\ref{sec:wsnapshot-opt-rate} --
it can be seen for the 6 hour observation the raw compute loads are
very similar. 

\begin{maxima}[]

realonly:true;

wstot: CompRate = (wsFFTTot+wsImCorrectTot)/tobs + wsGridRate;

algsys(float([diff(ev(wstot, append([NAA=9, Nfacet=1, tobs=(6*3600`s)], SKA1Low) ) / (1 `(Ops/s)), wsNWPlanes)]), [wsNWPlanes]);

float(ev(rhs(wstot), append([NAA=9, Nfacet=1, tobs=(6*3600`s), wsNWPlanes =234], SKA1Low)));

algsys(float([diff(ev(wstot, append([NAA=9, Nfacet=1, tobs=(6*3600`s)], SKA1Mid) ) / (1 `(Ops/s)), wsNWPlanes)]), [wsNWPlanes]);
float(ev(rhs(wstot), append([NAA=9, Nfacet=1, tobs=(6*3600`s), wsNWPlanes =174], SKA1Mid)));

algsys(float([diff(ev(wstot, append([NAA=9, Nfacet=1, tobs=(6*3600`s)], SKA1Survey) ) / (1 `(Ops/s)), wsNWPlanes)]), [wsNWPlanes]);
float(ev(rhs(wstot), append([NAA=9, Nfacet=1, tobs=(6*3600`s), wsNWPlanes =60], SKA1Survey)));



\maximaoutput*
\m  \mathbf{true} \\
\m  \left({{16\,\left(N_{\rm a}-1\right)\,N_{\rm a}\,N_{\rm beam}\,N_{\rm f}\,N_{\rm facet}^2\,\left(\mathrm{qty}\left({{16\,\mathrm{qty}\left({{3.48 \times 10^{-6}\,B_{\rm max}\,\mathrm{twsnap}\,\sqrt{{{B_{\rm max}\,\mathrm{twsnap}}\over{\lambda}}}}\over{D_{\rm s}}}\;{{1}\over{\mathrm{s}^{{{3}\over{2}}}}}+{{3.3 \times 10^{-10}\,B_{\rm max}^2\,\mathrm{twsnap}^2}\over{D_{\rm s}^2}}\;{{1}\over{\mathrm{s}^2}}\right)\,\lambda^2}\over{D_{\rm s}^2}}\right)+N_{\rm AA}^2\right)}\over{t_{\rm dump}}}+{{320\,B_{\rm max}^2\,N_{\rm beam}\,N_{\rm f}\,\log \left({{16\,B_{\rm max}^2}\over{D_{\rm s}^2\,N_{\rm facet}^2}}\right)}\over{\log 2\,D_{\rm s}^2\,\mathrm{twsnap}}}+{{3200\,B_{\rm max}^2\,N_{\rm beam}\,N_{\rm f}}\over{D_{\rm s}^2\,\mathrm{twsnap}}}\right)\;\mathrm{Ops}=\left({{{{320\,B_{\rm max}^2\,N_{\rm beam}\,N_{\rm f}\,\log \left({{16\,B_{\rm max}^2}\over{D_{\rm s}^2\,N_{\rm facet}^2}}\right)\,\mathrm{wsNWPlanes}}\over{\log 2\,D_{\rm s}^2}}+{{3200\,B_{\rm max}^2\,N_{\rm beam}\,N_{\rm f}\,\mathrm{wsNWPlanes}}\over{D_{\rm s}^2}}}\over{\mathrm{tobs}}}+{{8\,\left(N_{\rm a}-1\right)\,N_{\rm a}\,N_{\rm beam}\,N_{\rm f}\,N_{\rm facet}^2\,\left({{B_{\rm max}^2}\over{4\,D_{\rm s}^2\,N_{\rm facet}^2\,\mathrm{wsNWPlanes}^2}}+N_{\rm AA}^2\right)}\over{t_{\rm dump}}}\right)\;\mathrm{Ops} \\
\m  \left[ \left[ \mathrm{wsNWPlanes}=230. \right]  \right] \\
\m  6.9 \times 10^{+14}\;{{\mathrm{Ops}}\over{\mathrm{s}}} \\
\m  \left[ \left[ \mathrm{wsNWPlanes}=170. \right]  \right] \\
\m  7.4 \times 10^{+15}\;{{\mathrm{Ops}}\over{\mathrm{s}}} \\
\m  \left[ \left[ \mathrm{wsNWPlanes}=60. \right]  \right] \\
\m  5.31 \times 10^{+15}\;{{\mathrm{Ops}}\over{\mathrm{s}}} \\
\end{maxima}



\bibliographystyle{mn2eurl} 
\bibliography{ska.bib}

\end{document}

\end{document}

% Local Variables:
% LaTeX-command: "latex -shell-escape"
% End:
